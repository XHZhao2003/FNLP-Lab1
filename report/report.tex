\documentclass[12pt,letterpaper]{article}
\usepackage[UTF8]{ctex}
\usepackage{fullpage}
\usepackage[top=2cm, bottom=4.5cm, left=2.5cm, right=2.5cm]{geometry}
\usepackage{amsmath,amsthm,amsfonts,amssymb,amscd}
\usepackage{lastpage}
\usepackage{enumerate}
\usepackage{fancyhdr}
\usepackage{mathrsfs}
\usepackage{xcolor}
\usepackage{graphicx}
\usepackage{listings}
\usepackage{hyperref}
\usepackage{dot2texi}
\usepackage{tikz}
\usepackage{float}
\usepackage[pdf]{graphviz}
\usepackage{indentfirst} 
\usetikzlibrary{automata,shapes,arrows} 

\hypersetup{%
  colorlinks=true,
  linkcolor=blue,
  linkbordercolor={0 0 1}
}
 
\renewcommand\lstlistingname{Algorithm}
\renewcommand\lstlistlistingname{Algorithms}
\def\lstlistingautorefname{Alg.}

\lstdefinestyle{Matlab}{
    language        = matlab,
    frame           = lines, 
    basicstyle      = \footnotesize,
    keywordstyle    = \color{blue},
    stringstyle     = \color{green},
    commentstyle    = \color{red}\ttfamily
}
\setlength{\parindent}{0.0in}
\setlength{\parskip}{0.05in}
\setlength{\parindent}{2em}

% Edit these as appropriate
\newcommand\course{FNLP}
\newcommand\hwnumber{1}                  % <-- homework number
\newcommand\name{赵潇晗}                 % <-- Name
\newcommand\ID{2100013129}           % <-- ID

\pagestyle{fancyplain}
\headheight 35pt
\lhead{\name\\\ID}                 
\chead{\textbf{\Large Homework \hwnumber}}
\rhead{\course \\ \today}
\lfoot{}
\cfoot{}
\rfoot{\small\thepage}
\headsep 1.5em


\begin{document}

Lab1使用了Loglinear模型和TFIDF特征完成了给定的文本分类任务。
本实验报告解释了在Lab1中所实现的具体方法、进行实验和参数调优的细节、以及最终的性能评价。

\section{方法}
\subsection{数据清洗}
任务给出了训练与测试的新闻数据集。首先对每条数据的标题与内容进行分词和原形处理。
这个过程中借助了NLTK工具包。分词前清除了数据集中带有的分隔符(例如,"/")。
原形处理需要借助词性标记,因此这一过程使用了NLTK提供的POS标注工具,这一工具利用了基于WordNet的模型进行标注。

同时,还在训练集上统计了词汇表,以及词汇表中每个词出现的次数,供后续特征提取使用。

\subsection{特征提取}
基于训练集上统计的词汇表信息,提取新闻的TFIDF词频特征。

首先,在词汇表中选取出现次数多于500的词,作为unigram。
然后,排除其中不适合被利用进行特征提取的unigram。
其中一部分通过NLTK工具包下wordnet的stoplist模块排除,这部分主要是没有区分意义的代词,冠词等。
另一部分则是通过手动去除,例如人名,媒体信息(例如普遍出现的Reuters),数字,被意外分开的连接词前缀(例如 ex, anti)。

这样产生了1024个可用的unigram,记为$U = \{u_i\}_{i=1}^{1024}$。
对于$u_i$,记$N=120000$为样本总数,标题与内容包含$u_i$的训练集样本个数为 $m_i$。
由此得到逆文档频率 $$\text{IDF}_i = \log_{10}\frac{N}{m_i}$$

对于一条给定的新闻数据实例 $I=(\text{Label}, \text{title}, \text{description})$,
统计其标题与内容中每个unigram的出现次数 $t_i$。
由此得到词频率 $$ \text{TF}_i = 1 + \log_{10} t_i $$

$I$的TFIDF特征定义为 $F = [f_i]_{i=1}^{1024},\  f_i=\text{TF}_i \times \text{IDF}_i$。
其特征长度为1024。

按照课程中的定义方式,上述TFIDF特征应定义为 $F=[f_{i1}, f_{i2}, f_{i3}, f_{i4}]_{i=1}^{1024}$,
其中 $f_{i1}=f_{i2}=f_{i3}=f_{i4}$。其长度为4096。
为了实现的方便,Lab1使用了前一种定义,而四个类别分别设置模型参数。
显然两种方式是等价的,前一种有助于减少运算量,并更容易实现。

\subsection{Loglinear模型}
模型的参数为 $W=\{w_{ij}\}_{1024 \times 4}$。
对于样本 $I$ 及其特征 $F=[f_i]_{i=1}^{1024}$,模型预测其属于第$c$类的概率为
$$p(c|F) = \frac{\exp{q_c}}{\sum_{j=1}^4 \exp{q_j}}, c=1,2,3,4$$
$$q_j = \sum_{j=1}^{1024} w_{ij} \cdot f_i,\  j=1,2,3,4$$

\subsection{性能度量}
使用准确率与Macro-F1度量模型在目标数据集上的表现。
记$T=\{t_{ij}\}_{4 \times 4}$,其中$t_{ij}$代表标签为$i$并被模型预测为$j$的样本个数。
准确率定义为 
$$\text{Accuracy} = \frac{\sum_{i=1}^4 t_{ii}}{\sum_{i=1}^4\sum_{j=1}^4 t_{ij}}$$

Macro-F1为,
$$\text{Macro-F1} = \frac{1}{4}\sum_{i=1}^4 \text{F1}_i$$
$$\text{F1}_i = \frac{2 P_i R_i}{P_i + R_i}, P_i = \frac{t_{ii}}{\sum_{a=1}^4 t_{ai}}, R_i = \frac{t_{ii}}{\Sigma_{a=1}^4 t_{ia}}$$


\section{实验}
\subsection{训练方法}
Loglinear模型的参数初始化是随机的,使用batch梯度下降方法训练模型参数。使用交叉熵损失函数进行训练,
并使用了L2正则化项,即
$$L = \sum_{i=1}^B \ - \log p_{t_i} + \lambda || W ||_2$$ 
其中 $p_{t_i}$为模型预测其属于所属类别的概率。

Lab1没有实现自动微分过程,也没有使用自动微分工具包。由于训练函数是固定的,其负梯度为
$$ - \frac{\partial L}{\partial w_{nc}} = \sum_{i: t_i = c} f_n - \sum_{i=1}^B f_n p_c  - \lambda w_{nc}, \quad n=1,\cdots,1024, \ c=1,2,3,4$$

因此训练时直接按照这一负梯度更新参数。

为了优化训练性能,Lab1实现了学习率指数衰减方法。在每一epoch训练结束后以固定衰减率 $\gamma$ 减小学习率。
这有助于使模型在训练前期较快优化性能,并在训练后期拥有较小的学习率,防止出现性能震荡现象。

\subsection{训练参数}
训练过程所需的全部参数包括,训练epoch数$M$,Batch大小$B$,学习率$\alpha$,正则化系数$\lambda$,衰减率$\gamma$。

Lab1将数据集分为3部分,训练集,测试集,验证集。大小分别为 112400, 7600, 7600。
使用训练集训练模型,在验证集上验证性能,并据此性能调优上述参数。

具体来说,初步调优固定$B=32,\ M=50,\ \alpha=0.01,\ \gamma=0.99$。
在此基础上,在$\{0.01, 0.1, 1, 10\}$范围内尝试正则化系数,并最终选取 $\lambda=$


\section{结果}
在测试集上测试模型性能。
考虑到模型初始化,以及训练过程中的随机性,以相同参数训练模型10次并测试性能。


其中一次的训练结果保留在 model 目录下。

\end{document}
